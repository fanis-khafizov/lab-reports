\documentclass[12pt]{article}    
\usepackage{ucs} 
\usepackage[utf8x]{inputenc}
\usepackage[russian]{babel}  
\usepackage{float}
\usepackage{amsmath}
\usepackage{autonum}
\title{Лабораторная работа №5\\
	Определение удельного заряда электрона}
\author{Хафизов Фанис}
\usepackage[pdftex]{graphicx}
\usepackage{multirow}
\begin{document}
	\begin{figure}
		\centering
		\includegraphics[width=0.3\linewidth]{logo}
	\end{figure}
	\maketitle
	\newpage
	\section{Цель работы}
	Определить удельный заряд электрона с помощью катушек Гельмгольца.
	\section{Оборудование}
	Узкая электронно-лучевая трубка, катушки Гельмгольца, источник напряжения 300 В, регулируемый источник напряжения 0..300 В, цифровые мультиметры (2 шт), соединительные провода.
	\section{Порядок действий}
	\begin{enumerate}
		\item Соберем экспериментальную установку.
		\item Зафиксируем значение напряжения $U=150$ В и снимем зависимость $r(I)$ радиуса пучка электронов от силы тока в катушках.
		\item Зафиксируем силу тока в катушках равную $I=1,5$ А и снимем зависимость $r(U)$ циклотронного радиуса электронов от ускоряющего напряжения.
	\end{enumerate}
	\section{Теоретическая зависимость}
	\[
	r(I,U)=\frac{R}{\mu_0n}\sqrt{\frac{125}{32\gamma}}\frac{\sqrt{U}}{I}
	\label{eq}
	\]
	Следовательно, зависимости $r(1/I)$ и $r(\sqrt{U})$ линейны.
	\section{Таблицы данных и графики}
	\begin{table}[H]
		\centering
		\begin{tabular}{|l|r|r|}
			\hline
			$r$, см & \multicolumn{1}{l|}{$I$, А} & \multicolumn{1}{l|}{$1/I$, 1/А} \\ \hline
			2,0     & 3,37                        & 0,297                           \\ \hline
			2,5     & 2,67                        & 0,375                           \\ \hline
			3,0     & 2,2                         & 0,455                           \\ \hline
			3,5     & 1,89                        & 0,529                           \\ \hline
			4,0     & 1,62                        & 0,617                           \\ \hline
			4,5     & 1,43                        & 0,699                           \\ \hline
			5,0     & 1,27                        & 0,787                           \\ \hline
		\end{tabular}
		\caption{Зависимости $r(I)$ и $r(1/I)$}
	\end{table}
	\begin{table}[H]
		\centering
		\begin{tabular}{|l|r|r|}
			\hline
			$r$, см & \multicolumn{1}{l|}{$U$, В} & \multicolumn{1}{l|}{$\sqrt{U}$, $\sqrt{B}$} \\ \hline
			2,0     & 49                          & 7,0                                         \\ \hline
			2,5     & 54                          & 7,3                                         \\ \hline
			3,0     & 63                          & 7,9                                         \\ \hline
			3,5     & 90                          & 9,5                                         \\ \hline
			4,0     & 129                         & 11,4                                        \\ \hline
			4,5     & 169                         & 13,0                                        \\ \hline
			5,0     & 218                         & 14,8                                        \\ \hline
		\end{tabular}
		\caption{Зависимости $r(U)$ и $r(\sqrt{U})$}
	\end{table}
	\begin{figure}[H]
		\centering
		\includegraphics[width=0.9\linewidth]{graph1}
		\caption{График зависимости $r(1/I)$}
	\end{figure}
	\begin{figure}[H]
		\centering
		\includegraphics[width=0.9\linewidth]{graph2}
		\caption{График зависимости $r(\sqrt{U})$}
	\end{figure}
	Как можно заметить, в графике зависимости $r(\sqrt{U})$ первые две точки выбиваются из всей серии, что можно объяснить нестабильностью пучка электронов при малых значениях ускоряющего напряжения. Построим этот график без этих двух точек.
	\begin{figure}[H]
		\centering
		\includegraphics[width=0.9\linewidth]{graph3}
		\caption{График зависимости $r(\sqrt{U})$ без 2 точек}
	\end{figure}
	\section{Расчеты}
	Из коэффициента наклона графика $r(1/I)$:\\
	$\displaystyle\alpha = 6,12$ см$\cdot$А$=6,12\cdot10^{-2}$м$\cdot$А\\
	Из формулы (\ref{eq}):\\
	$\displaystyle\alpha = \frac{R}{\mu_0n}\sqrt{\frac{125}{32\gamma_1}}\sqrt{U}$\\
	$\displaystyle\gamma_1=\frac{125}{32}\left(\frac{R}{\alpha\mu_0n}\right)^2U=\frac{125}{32}\left(\frac{0,2}{6,12\cdot10^{-2}\cdot1,26\cdot10^{-6}\cdot154}\right)^2\cdot 150=\\=1,66\cdot10^{-11}$Кл/кг\\
	Коэффициент наклона графика $r(\sqrt{U})$:\\
	$\displaystyle\beta=0,291$ см$/\sqrt{B}=0,291\cdot10^{-2}$ м$/\sqrt{B}$\\
	Из формулы (\ref{eq}):\\
	$\displaystyle\beta=\frac{R}{\mu_0nI}\sqrt{\frac{125}{32\gamma_2}}$\\
	$\displaystyle\gamma_2=\left( \frac{R}{\mu_0nI\beta}\right)^2\cdot\frac{125}{32}=\left(\frac{0,2}{1,26\cdot10^{-6}\cdot154\cdot1,5\cdot0,291\cdot10^{-2}}\right)^2\cdot\frac{125}{32}=\\=2,17\cdot10^{-11}$ Кл/кг\\\\
	Оценка погрешностей:\\
	$\displaystyle\Delta\gamma_1=\varepsilon_{\gamma1}\cdot\gamma_1=(2\varepsilon_r+2\varepsilon_I+\varepsilon_U)\cdot\gamma_1=\left(2\frac{\Delta r}{r}+2\frac{\Delta I}{I}+\frac{\Delta U}{U}\right)\cdot\gamma_1=\left(2\frac{0,25}{2,0}+2\frac{0,01}{1,27}+\frac{0,5}{150}\right)\cdot1,66\cdot10^{-11}=0,45\cdot10^{-11}$ Кл/кг\\
	$\displaystyle\Delta\gamma_2=\varepsilon_{\gamma2}\cdot\gamma_2=(2\varepsilon_r+2\varepsilon_I+\varepsilon_U)\cdot\gamma_1=\left(2\frac{\Delta r}{r}+2\frac{\Delta I}{I}+\frac{\Delta U}{U}\right)\cdot\gamma_1=\left(2\frac{0,25}{2,0}+2\frac{0,01}{1,5}+\frac{0,5}{63}\right)\cdot2,17\cdot10^{-11}=0,59\cdot10^{-11}$ Кл/кг\\
	\section{Результаты}
	Табличное значение удельного заряда электрона:\\
	$\displaystyle\gamma=1,76\cdot10^{-11}$ Кл/кг\\
	Значения удельного заряда электрона, полученные в ходе эксперимента:\\
	$\displaystyle\gamma_1=\left(1,66\pm0,45\right)\cdot10^{-11}$ Кл/кг\\
	$\displaystyle\gamma_2=\left(2,17\pm0,59\right)\cdot10^{-11}$ Кл/кг\\
	$\varepsilon_{\gamma1}=28\%$\\
	$\varepsilon_{\gamma2}=28\%$\\
	Первый метод дал более точное значение удельного заряда электрона, однако, с учетом погрешности табличное значение попадает в ворота обоих методов.\\
	Оцененные погрешности совпали, но в первом опыте мы получили более близкое к истинному значение, так как было зафиксировано ускоряющее напряжение, и изменялась лишь индукция магнитного поля. Во втором же случае менялось ускоряющее напряжение, что могло негативно сказаться на стабильности пучка элуктронов.
	\section{Выводы}
	В ходе проведения эксперимента нам удалось измерить удельный заряд электрона с достаточно высокой точностью. В обоих опытах относительная погрешность равна $28\%$. Она настолько большая из-за неточного метода измерения радиуса траектории (было предложено делить отрезок известной длины пополам на глаз), что сказалось на точности самого ответа. Чтобы уменьшить погрешность можно закрепить более точный измерительный прибор внутри трубки, например, линейку.
\end{document}
